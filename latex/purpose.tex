%% Chapter purpose

The Large Synpotic Survey Telescope (LSST) Data Management (DM) team
have developed a large set of software modules for astronomical image
reduction and analysis.  The code is open-source and designed to be
generally useful for all sorts of imaging surveys.  Development has
reached the point where it may be useful for scientists inside or
outside the LSST project to use the software for general work.  This
guide is intended for scientists who have not written any of the code
and who are unfamiliar with the tools upon which the LSST software
stack is built.  It contains pointers to resources for those who wish
to drill down very deeply, but aims to be reasonably self-contained
for newbies.

In LSST world, we write number-crunching code in C++ and string
together the number-crunching steps with Python.  This allows us to
take advantage of the speed of C++ when necessary, and the ease of
Python otherwise.  You may be familiar with SciPy/NumPy, which give
you the ease of Python\footnote{If you are not familiar with Python,
  you should become familiar with it, even if you have no plans to
  work with LSST software.  It's extremely useful.  See XXX for a good
  tutorial.} while enabling the speed of C++ ``under the hood.''  You
can think of the LSST software as providing astronomy-specific
functionality ``under the hood'' in a similar way, but you will not be
quite as insulated from the C++ details as you would be with
SciPy/NumPy.  This document assumes that you will be writing Python
rather than C++, but you may need to refer to the C++ documentation to
see what features are available.  We will guide you through that
process.

