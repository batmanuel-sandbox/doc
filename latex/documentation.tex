% documentation

\renewcommand{\LaTeX}{LaTeX\xspace}

\newcommand{\tttex}{\texttt{.tex}\xspace}
\newcommand{\ttlsstpackages}{\texttt{lsstPackages.tex}\xspace}
\newcommand{\ttinput}{\texttt{\textbackslash input}\xspace}
\newcommand{\ttlsstmanual}{\texttt{lsstManual.tex}\xspace}
\newcommand{\ttdevenvdoc}{\texttt{devenv/doc}\xspace}
\newcommand{\tteups}{\texttt{eups}\xspace}
\newcommand{\ttsetup}{\texttt{setup}\xspace}
\newcommand{\ttscons}{\texttt{scons}\xspace}
\newcommand{\ttlsstinput}{\texttt{\textbackslash lsstinput}\xspace}
\newcommand{\ttlatextohtml}{\texttt{latex2html}\xspace}

\newcommand{\ttdoxygenconf}{\texttt{doxgyen.conf}\xspace}

\newcommand{\ttpackage}{\texttt{package.tex}\xspace}

\newcommand{\todo}[1]{{\bfseries\large TODO: #1}}

% latex, html and doxygen
\section{Documentation Overview\label{sec-doc-overview}}

LSST software is documented in two fundamental ways:

\begin{description}
  
\item[The LSST Developer's Manual] This is a single document, written
  in \LaTeX, and is intended to be a `manual' in the traditional sense
  of the word.  It's available online in HTML format, and also as a
  PDF document. You're currently reading the Developer's Manual!  The
  HTML version of the current manual can be found online at \todo{add
    location of html manual}, and the PDF version is availabe at:
  \todo{add location of pdf manual}.

\item[The Doxygen source code documentation] This is a source code
  reference, containing summaries of the various classes and
  functions, and their associated parameters (ie. arguments, return
  values, etc.).  It is an HTML (online) reference which allows a
  developer to look up the application programming interface (API) for
  a given class or function, and to link quickly to other related
  classes and parameters.  Doxygen is machine generated directly from
  source code, with additional information (variable definitions,
  brief summaries, etc) taken from specific comments included by the
  programmer. Because of the way it is used (`browsing' source code),
  it is available only in HTML.  The current LSST Doxygen
  documentation can be found at:
  \url{http://dev.lsstcorp.org/doxygen/trunk/}.  For more information
  about Doxygen software, see \url{http://www.doxygen.org}.
    
\end{description}

Each of these forms of documentation is available for individual
products, and for the entire LSST project.  Here, we'll describe the
details of the LSST documentation system.


% main doc product
\section{The Main Documentation Product, \ttdevenvdoc\label{sec-maindoc}}

The main documentation product lives in \ttdevenvdoc, and much
of the documentation contained in the Developer's Manual can be found
here.  However, the parts of the manual dealing with individual
products are stored with the relevant products, and are pulled into
the manual when it is compiled.

Conversely, there is very little Doxygen documentation stored in
\ttdevenvdoc.  When built, the product collects Doxygen
configuration files from all available (\ttsetup) products and creates a
single comprehensive Doxygen reference.

The documentation product differs in structure from other standard
LSST products, containing only \texttt{doxygen/} and \texttt{latex/}
directories.

\subsection{The Main \LaTeX Documentation\label{sec-mainlatex}}

The highlights of the \texttt{latex/} directory include:

\begin{description}
  \item [\ttlsstmanual] This is the central \LaTeX document that gets
    built.  However, it contains only \ttinput
    statements for the various chapters.
  \item [\ttlsstpackages] This is a machine-generated file which
    \ttinput s the \LaTeX{} documentation for the
    individual LSST products.  Its creation is handled in the
    \texttt{latex/SConscript} file, where the absolution paths to the
    currently setup product documentation are prepended to the
    \ttinput filename.
  \item [{\itshape everything-else}\tttex] The remaining
    *\tttex files contain the chapters which are imported
    into \ttlsstmanual.
  \item[\texttt{figures/}] This subdirectory contains any figures
    used in the manual.
  \item[\texttt{htmlDir/}] This subdirectory contains an HTML
    version of the manual, generated with \ttlatextohtml.  The
    directory and its contents will not exist before \ttscons is
    run.
\end{description}

{\bfseries If you'd like to edit existing documentation}, the rest of
this section is unlikely to be of interest (or use) to you.  All you
have to do is find the appropriate \tttex file and make your changes.
Then run \ttscons to build the new manual.

{\bfseries If you'd like to add a new chapter}, simply put your
chapter in an appropriately named \tttex file and add an \ttinput
statement to \ttlsstmanual to import it.

{\bfseries If you'd like to understand how \LaTeX from the individual
  products gets imported} into the top-level manual, read on.

The importing of the individual product documentation requires a bit
of explanation.  The system is designed such that the final document
which gets built (by scons) will pull in the individual product
documentation from any products which were \ttsetup (using the
\tteups `\ttsetup' command).

As alluded to in the file explanations above, these \tttex files for
each product are imported in the \ttlsstpackages file.
During the build, the \texttt{latex/SConscript} code determines which
products are \ttsetup (and {\itshape where}) and generates the
\ttlsstpackages file with an \ttinput statement for each.  However, if
you look in the \ttlsstpackages file, that's not what you'll see.

It's entirely possible that each imported \tttex file will also
contain \ttinput statements for files in its own directory tree.  To
deal with this, {\itshape all individual products must only use our
  own \ttlsstinput command!}  The \ttlsstinput command is a thin
wrapper for the regular \LaTeX \ttinput command.  For each imported
product, the \ttlsstpackages file contains two lines: the first renews
the \ttlsstinput command to prepend the absolution path for the
product in question, the second then \ttlsstinput's the product.

When \ttscons is run, the \LaTeX is compiled and
\texttt{lsstManual.pdf} is created in the root directory of
\ttdevenvdoc.  At the same time, an HTML version of the manual will be
created in \texttt{latex/htmlDir}.

\iffalse
%
% Doxygen is now built using the devenv/lsstDoxygen package
%
\subsection{The Main Doxygen Documentation\label{sec-maindoxygen}}

Doxygen builds its documentation by digesting the source code for a
project.  The construction of the LSST doxygen is controlled by a
\ttdoxygenconf file which identifies the locations (directories) of the
source code to be scanned.  The \ttdevenvdoc \texttt{doxygen/}
directory contains no \ttdoxygenconf file, but rather a python script
\texttt{writeDoxyConfig.py} which is executed when \ttscons is run.
As you might suspect, \texttt{writeDoxyConfig.py} scans all available
\ttdoxygenconf files (ie. those \ttsetup via \tteups) and generates a
master \ttdoxygenconf.  This machine-generated file is then used to
build the top-level Doxygen.  This happens when \ttscons is run and is
handled in the \texttt{doxygen/SConscript} file.

When \ttscons is run, the Doxygen output will be created in
\texttt{doxygen/htmlDir}.
\fi

% individual product documentation
\section{Documenting an Individual Product\label{sec-productdoc}}

The documentation for each individual product is within the product's
\texttt{doc/} directory.  There, you'll find a \ttdoxygenconf file
(which you should have no need to edit), and a \texttt{latex/}
subdirectory.


\subsection{Standard LSST Doxygen Coding Practices\label{sec-doxypractices}}

The LSST Coding practices are described in detail on the LSST Trac
wiki.  To avoid duplication (and inevitable inconsistency), we simply
refer you to:
\url{http://dev.lsstcorp.org/trac/wiki/DocumentationStandards}.

When \ttscons is run, the Doxygen HTML will be written to \texttt{doc/htmlDir/}.


\subsection{Contributing to the Manual Entry for a Product\label{sec-productlatex}}

The \texttt{doc/latex/} directory should contain at least two \tttex
files: one named after the product itself
(eg. \texttt{meas\_astrom.tex}, \texttt{afw.tex}), and one named
\ttpackage.  {\bfseries Documentation should go in the \ttpackage
  file}.  It is the \ttpackage file which will be imported into the
top-level manual when \ttdevenvdoc is built.  The \tttex file bearing
the name of the product is just a thin wrapper which \ttinput's the
\ttpackage file to build a local PDF copy of the manual for the
product.

If you'd like to put your section/subsection in a separate \tttex
file, feel free to do so.  However, {\bfseries in order for the
  documentation to be properly imported into the top-level manual, you
  must use the \ttlsstinput command rather than the normal \LaTeX
  \ttinput command!}

Not all products are documented in the \LaTeX manual, so you may find
that there's no \texttt{doc/latex/} directory in a some products.  If
you'd like to add one, please do!

Here's a checklist of things to do to add latex documentation to a product:

\begin{enumerate}
  \item Create the \texttt{doc/latex/} directory.
  \item Create \texttt{doc/latex/{\itshape myproduct}.tex} (adapt it
    from a product which already has documentation).
  \item Create \texttt{doc/latex/package.tex} with your contribution to the manual.
  \item Create \texttt{doc/latex/SConscript} to build it (again, adapt
    it from the \texttt{SConscript} file in a product which already
    has documentation).
  \item Edit \texttt{SConstruct} to call the new SConscript.
\end{enumerate}

