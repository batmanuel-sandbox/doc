% howto

We've seen that when browsing the online documentation, it's quite
easy to look at a C++ definition and see how to use it in Python and
what it's useful for.  It's much harder to go the other way: given a
goal, what are the classes that enable that goal?  This chapter
provides answers for common tasks. If you encounter a task you think
should be listed here, just send it to us and we'd be happy to include
it.

\section{python}

\subsection{Importing an LSST package in python}

on the surface, importing an LSST package is just like importing any other python package:
\begin{verbatim}
import lsst.afw.image
import lsst.meas.algorithms
\end{verbatim}

However, if you look a little harder you'll see that there's something odd going on; \verb|lsst/afw/image| and
\verb|lsst/meas/algorithms| are not subdirectories of the same parent (they refer to
\verb|$AFW_DIR/python/lsst/afw/image| and \hfil\break\verb|$MEAS_ALGORITHMS_DIR/python/lsst/meas/algorithms| respectively);
that's not something that python's \verb|import| command understands.  What's more, there is no file
\verb|$AFW_DIR/python/lsst/__init__.py|.

The trick is that there \textit{is} a file \verb|$BASE_DIR/python/lsst/__init__.py|, and it installs a custom
package importer into \verb|sys.meta_path| (this requires python 2.5; the details are in PEP 302 if you care)\footnote{There's similar code in \texttt{sconsUtils} too, as it too has an lsst directory that needs to be imported}.
This importer is not restricted to loading a sub-module from a subdirectory of the directory where its parent
module was loaded from.  Thus, if your sys.path contains multiple directories that contain an \verb|lsst|
module, then a submodule \verb|lsst.foo| can appear below any of those directories.

