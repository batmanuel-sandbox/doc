% overview

Having demonstrated how easy it is to do powerful things with this
software, let's take a brief look at each of the available packages
and what they can do.  Later in this book, each package will star in
its own more detailed chapter.

\section{Online documentation}

It will be useful to browse the online documentation while reading
through this chapter.  As the C++ programmers write their code, they
build in a certain amount of documentation through a system called
Doxygen.  Doxygen generates a web page for each package and class and
makes it relatively easy to click through.  For example, if a method
takes an argument of a type you don't recognize, you should be able to
click on it and get the definition of that type.

The most convenient version of the Doxygen docs is available online at
\url{http://www.astro.princeton.edu/~rhl/LSST/Doxygen/htmlDir/index.html}.
While the code is constantly changing, this website is not constantly
updated; however, it should be sufficient for most needs.  If necessary,
there is a way to generate the docs from the exact version of the code that
exists on your computer, but it won't be as nicely crosslinked as this
version; you will have to search each package separately, for example.

This section will describe how to navigate the online documentation,
but first we must discuss the concept of {\it namespace}.  In older
languages such as Fortran and C there is only one namespace; each
distinct function must have a distinct name.  This could become a
problem if you were using multiple third-party libraries and different
libraries happened to use the same name for different things,
resulting in a {\it name collision}.  In Python and C++, name
collisions are avoided by having multiple namespaces.  For example, in
Python the statement \texttt{import lsst.afw.image as afwImg} means
that there is an lsst namespace, within that an afw namespace, and
within that an image namespace; and to avoid the hassle of constantly
typing lsst.afw.image, we will define afwImg as a nickname.  So we can
refer to the constructor \texttt{lsst.afw.image.ImageF} as \texttt{afwimage.ImageF}
without worrying that there might also be an \texttt{ImageF} defined
in \texttt{some.other.package}; that \texttt{ImageF} would have to be
referred to as \texttt{some.other.package.ImageF}.  See
\url{http://docs.python.org/tutorial/classes.html#python-scopes-and-namespaces}
for a more detailed description.


\subsubsection{Navigating the maze:  The tabs}

Go ahead and click on
\url{http://www.astro.princeton.edu/~rhl/LSST/Doxygen/htmlDir/index.html}
if you haven't already.  The {\it Main Page} is titled {\it lsst::afw; the
LSST Application Framework} and is indeed afw-centric.  While afw is
the obvious starting place for newbies, keep in mind that there is a
{\it lot} of functionality in other namespaces, and it's {\it all}
documented here, despite the title.  The major difference is that afw
has a fair amount of readable documentation here which goes above and
beyond the standard ``here is the definition of class foo and here are
all of its constructors.''  The main page links directly to a handful
of useful writeups, such as {\it How to display images}.

The {\it Namespaces} and {\it Classes} tabs will be your workhorses.
Clicking on the Namespace tab yields an alphabetical list of
namespaces.  One quirk: some namespaces within afw are listed first,
because afw is first alphabetically; but others are listed under
lsst::afw. (Note: the '::' separator is C++'s equivalent of Python's
'.' separator for namespaces.)  The reason for this is that some
namespaces were defined in Python and some in C++; Doxygen makes the
former appear as afw::xxxx while the latter appear as lsst::afw::xxxx.
Although of no real significance, it can be confusing: clicking on
afw::image yields a very boring page containing only testUtils and
utils, while clicking on lsst::afw::image yields the richly detailed
page you really wanted.  

Classes are the abstract definitions of what we previously referred to
as ``objects.'' For example, lsst::afw::image::ImageF is a class.
When we use the constructor lsst::afw::image::ImageF(filename), it
returns an object of that type.  Internally, classes contain data
members (for example, the size of the image), but these are generally
not accessible externally.  All external manipulations are supposed to
be done using the methods listed.  Note that if you go to the Classes
tab, you will {\it not} find lsst::afw::image::ImageF. Rather, you
will find \texttt{lsst::afw::image::Image< PixelT >}.  This is
because Image is a {\it templated} class.  The bit in angle brackets
is a stand-in for F, D, or I depending on whether the image is float,
double, or int.  Templates allow C++ programmers to write the code
once, and have the compiler ``expand the template'' for them, rather
than write essentially the same code over again for each new type.

Go ahead and click on the \texttt{lsst::afw::image::Image< PixelT >}
you find listed in the Namespaces tab.  As promised in the previous
chapter, you will find all the gory detail of all the different
constructors and operators that have been defined.  (Note: rhs and lhs
stand for right-hand side and left-hand side respectively.)  You will
often see the same operator defined multiple times.  For example, '-='
is defined once for a scalar rhs and once for an image rhs.  This is
what allows you, the Python programmer, to put either type on the rhs
and have it just work automatically.

A further word on navigation: at the top of the page you see the
clickable text \texttt{lsst::afw::image::Image}.  Note that {\it each
  item separated by :: is separately clickable}.  So if you want to
see what else is in the \texttt{lsst::afw::image} namespace, click on
the first \texttt{image}; to see what else is in the
\texttt{lsst::afw} namespace, click on \texttt{afw}, etc.  This is
often a useful way to get where you want, if you find the
semi-alphabetical listing of namespaces in the Namespaces tab too
confusing.

For completeness, we also describe the less oft-used tabs.  The {\it
  Related Pages} tab contains a fairly long list of pages which are
meant to be tutorials or introductions in the style of those linked to
from the Main Page.  But many of these pages are blank, and some of
the non-blank ones have syntax errors or are out of date.  These may
still be useful as a rough guide to how the developers intended the
code to be used, but do not expect them to work verbatim.  Ignore the
{\it Modules} tab; ``module'' is not a well-defined concept as is
namespace or class, and here it contains an odd collection of things.
The {\it Files} and {\it Directories} tabs list all source code files and
directories; while straightforward, these are not immensely useful to
the beginner.  But note that everything is clickable, so if you must
hunt through source code files and directories, doing it here may be
more convenient than doing it at your Unix prompt.  The {\it Examples}
tab will be useful mainly for C++ programmers.

Ready?  Fasten your seat belts for a quick tour through LSST land!

\section{lsst::afw}

The application framework defines basic behavior of images, background
models, source detections, and the like, as well as supporting actors
such as bounding boxes. At the highest level (\url{http://www.astro.princeton.edu/~rhl/LSST/Doxygen/htmlDir/namespacelsst_1_1afw.html}\footnote{Not
  to be confused with
  \url{http://www.astro.princeton.edu/~rhl/LSST/Doxygen/htmlDir/namespaceafw.html},
  as mentioned above.}) we see only eight items, the namespaces
\texttt{cameraGeom, coord, detection, display, formatters, geom,
  image}, and \texttt{math}.  Rest assured, there is a lot within!
The really useful namespaces are \texttt{detection, display}, and
\texttt{image}.  We'll start with \texttt{image} because it's so
fundamental.

%\footnote{For
%  completeness, scons and SconsUtils are for compiling the C++ code;
%  cameraGeom defines things like amps, ccds, and rafts; and geom
%  defines ellipses.}  

\subsection{lsst::afw::image}

Defines basic behavior of images.

\subsection{lsst::afw::detection}

Tools for detecting sources in images.

\subsection{lsst::afw::display}

Tools for displaying images and (presumably) associated helpers like
bounding boxes.

\section{lsst::meas}

\section{lsst::ip}

\subsection{lsst::ip::isr}

\subsection{lsst::ip::diffim}
