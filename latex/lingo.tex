% lingo

%\section{Quick Summary of Object-Oriented Programming Lingo}

If you are unfamiliar with object-oriented programming (OOP), this is an
{\it extremely} brief summary, just enough to define some terms so
that you can start reading this manual fruitfully.  You are advised to
consult other references to gain a deeper understanding.

An {\it object} or a {\it class} is a software construct that contains
both data and algorithms (called {\it methods} for manipulating that
data.  (A key insight of OOP is that keeping data and algorithms
together makes software more robust and reusable.)  For example, the
{\it image} class contains data members such as number of pixels in
each dimension, and memory for containing the actual pixel values.
When the name of the class is invoked on the right-hand side (for
example, \texttt{im1 = afwImg.ImageF(`image1.fits')}), it serves as a
{\it constructor}, which allocates and initializes the data members
(or {\it instantiates} the object) and returns a reference to the
object.

The image class also contains methods such as multiplication by a
scalar and multiplication by another image. Obviously, the internal
workings of `` multiplication by a scalar'' and ``multiplication by
another image'' are quite different.  But we can present a simple
interface to people using our class, so that \texttt{im1 *= 2} invokes
`` multiplication by a scalar'' and \texttt{im1 *= im2} invokes
``multiplication by another image.''  This is known as {\it operator
  overloading}.

{\it Inheritance} is another big idea of OOP.  We can define {\it
  subclasses} which inherit the properties of their parent classes,
but tweak certain details.  As an example, if operation \texttt{foo}
can be done blindingly fast on square images but not on general
rectangular images, we would write the slow version of \texttt{foo}
for the \texttt{image} class, and then define a \texttt{squareimage}
subclass which inherits all the methods of an \texttt{image} but
overloads \texttt{foo} with its own blindingly fast algorithm.  This
is great for code reuse, because now if we add a new feature to (or
fix a bug in) the \texttt{image} class, it {\it automatically} also
gets incorporated into \texttt{squareimage} via inheritance.  We don't
need to touch the \texttt{squareimage} subclass.

Note that classes often have a variety of constructors.  For example,
\texttt{im1 = afwImg.ImageF(`image1.fits')} calls a constructor which
creates an in-memory \texttt{image} from a FITS file, but \texttt{a =
  afwImg.ImageF(10,10)} creates an in-memory \texttt{image} knowing
  only the desired dimensions (initializing the pixel values to zero).
  By looking at the types of the arguments starting with the first
  one, you will be able to deduce which method is actually called in
  any given case.  The error message you get when you use a
  constructor incorrectly is not very specific.  You might get a list
  of all possible constructors, which is not very helpful when there
  are dozens and you just want to know where your one little mistake
  is.  This is rather different from say, C, where the compiler knows
  what type the nth argument should be, and tells you quite
  specifically when you give it the wrong type.

