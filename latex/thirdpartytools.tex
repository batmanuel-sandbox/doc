% third party tools

The LSST C++ programmers themselves build upon numerous open-source
libraries and tools.  This appendix gives a brief overview of those
tools.  You do not have to read it thoroughly to make sense of the
rest of this document, but it may be useful to have some reference
point for the various terms that may appear when you install and run
the software.  LSST has blessed certain versions of these things as
the official LSST versions, which are installed in one big gulp by the
procedure described in Appendix~\ref{appendix-stackinstall}.  Thus you
do not have to follow the URLs in this appendix; they are for your
information only.

\section{SCons}

Think of SCons (\url{http://www.scons.org}) as the new ``make.''  (The
name comes from ``software construction.'')  Why don't we simply use
make?  Because SCons is more powerful; an SCons script {\it is} a
Python script, so it has the full power of Python when necessary, but
it automates the routine jobs.  That said, SCons will take some
getting used to if you are writing and building C++ code.  Those of
you sticking to Python will not need it.

\section{EUPS}

EUPS is the Extended Unix Product System.  It's a way to manage all of
the versions of all of the libraries and tools you may have on your
system.  For example, if you already have NumPy installed but it's not
the official LSST version, you need to set up various paths so that
when you run LSST software, it uses the expected version.  You do {\it
  not} have to delete your personal version, which you may need for
other projects.  (LSST software may in fact run with your
pre-installed version, but that is not guaranteed, and you will not
receive much help from the project if you encounter problems this
way.)  EUPS is designed to make this easy even in the presence of all
the various LSST and third-party packages with all their versions and
running on many different flavors of OS.  EUPS grew out of a similar
tool written by astronomers for the Sloan Digital Sky Survey, and is
now maintained by LSST.  Its home page appears to be
\url{http://dev.lsstcorp.org/trac/wiki/Eups}.

\section{svn}

As a user rather than an LSST developer, you may never have to use
svn, but you will likely hear references to it.  SVN (short for
``subversion'', \url{http://subversion.tigris.org}) is a modern
software repository system.  When developers check in new versions of
code, svn automatically saves the old version, records who checked in
the new version, can generate diffs when requested, etc.  And it's not
limited to code; the writers of the LSST Science Book used svn to
coordinate the contributions of hundreds of authors.  There is a
distinction between svn the software on one hand, and any particular
svn repository on the other.  LSST DM has one repository, the LSST
Science Book has another, and thousands of other software projects
have their own svn repositories.  An example of the kind of thing EUPS
should do is set up an environment variable that tells svn where the
LSST DM repository is, rather than force you to type it on the svn
command line.

Most large software projects make a public release by extracting a
particular snapshot of their svn repository and making it downloadable
in, say, one big gzipped tar file.  Users who download it thus do not
need an svn client; access to the bleeding-edge version of the code
would actually be harmful for many people.  LSST software is not quite
this publicly-accessible, but XXX

\section{Other third-party packages}

The LSST software stack makes use of many other third-party packages,
for example Boost, a set of general-purpose C++ libraries; cfitsio;
swig (a system for making C++ and Python, among other languages, work
together); Python and various Python modules; etc.  You probably don't
need to worry about any of these in any detail, but you will likely
see their names when you drill down a little deeper than this
document.

% Do we need these?
% \chapter{Glossary}
% \chapter{Index}
