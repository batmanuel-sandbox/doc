\section{Turning on debugging output (often image display)}

Many (but not all) of the LSST production scripts can be configured using the \verb|lsstDebug| package. \verb|lsstDebug| has a very negative view of the world; \verb|lsstDebug.Info(xyz).abc| returns \verb|False| for any value of\verb|xyz| and \verb|abc|.  By convention, packages call it as
\begin{verbatim}
import lsstDebug
display = lsstDebug.Info(__name__).display
\end{verbatim}
at the top of their main procedure (the choice of \verb|display| is common but not required); for example in \texttt{python/lsst/meas/utils/sourceDetection.py}. If you do nothing else, this sets \verb|display| to \verb|False|.

More interestingly, I can \verb|import| (or \verb|reload|) a file that loos something like:

\begin{verbatim}
import sys

import lsstDebug

def DebugInfo(name):
    di = lsstDebug.getInfo(name)        # N.b. lsstDebug.Info(name) would call us recursively
    if name in (
        "lsst.meas.utils.sourceDetection",
        "lsst.meas.astrom.determineWcs",
        ):
        di.display = 1

    return di

lsstDebug.Info = DebugInfo
\end{verbatim}

Suddenly, \verb|lsst.meas.utils.sourceDetection.display| is set to \verb|True| without my modifying its source or reloading it.

