% install

\section{How do I install all this stuff?}

The place to start is
\url{http://lsstdev.ncsa.uiuc.edu/trac/wiki/Installing}.  From here
you can jump to {\it Installing on Linux} or {\it Installing on
  Macintosh}.  Don't worry that the list of officially supported
platforms is short and probably doesn't include yours; in practice,
the stack should run on most Linux or Mac systems with only minor
tweaks, and most developers actually work on ``unsupported''
platforms!  The ``supported'' moniker indicates that when push comes
to shove, the highest priority for the project is that the code runs
on the big Linux clusters that will be used for production.  But if
you need help on an ``unsupported'' platform, ask around and you will
most likely get help.

The code does tend not to work on the absolute latest versions of
things (for example MacOS 10.6 and Linux versions which use gcc 4.4,
although there is an easy workaround for the latter).  Machines with
middle-aged OS's like MacOS 10.5 and Linux versions with gcc 4.3 are
going to work out of the box.

This author installed on Ubuntu 10.04, so we will go through the
process step-by-step for that OS.  Clicking on {\it Installing on
  Linux}, we get to a page which (at the bottom) lists the OS packages
which must be present before the stack can be installed; for
Debian/Ubuntu, we see g++, subversion, etc.  Copy and paste the
apt-get command for installing all these packages. It may vary
slightly for other versions of Ubuntu/Debian, as package names
sometimes change.

With that done, rejoin the OS-independent part of the install
instructions at
\url{http://dev.lsstcorp.org/GettingStarted.html#lsst_home}.  They ask
you to define a few environment variables, choose a directory to
install into (user space is probably best), grab the install script
from the given URL, and run it.  That's it.  It should take at least
60 minutes to download and compile everything. 

A warning: the install script may say "Installation complete" at the
end even if previous steps have failed!  Be sure to skim the output of
the install script to see if everything went ok.  In the case of
Ubuntu 10.04, afw fails to compile; this turns out to be a known issue
with a known, easy fix\footnote{So why doesn't the project
  automatically apply the fix?  Because it only occurs with gcc 4.4,
  which hasn't been officially adopted yet. By the time it is, a new
  version of the third-party library with the relevant bugfix will
  also be adopted.} described at
\url{http://dev.lsstcorp.org/trac/ticket/1124} (note: you may need a
login to see this part of the dev.lsstcorp.org site).  After applying
the fix, the install script runs to completion.

You may wish to scroll down the install instructions page for a brief
explanation of the subdirectory structure that's been installed.  This
explanation is most relevant for power users who will be running
multiple versions and platforms.

To use the software you must perform two configuration steps:
\begin{enumerate}
\item Tell Python where to find the LSST packages by
  running \\ \texttt{source /your/install/path/loadLSST.csh} (or loadLSST.sh
  for bash users).  You can test this by starting Python in
  interactive mode (just type \texttt{python}) and then telling Python
  \texttt{import lsst}.  If that doesn't fail with an error (sucess
  will be silent), then loadLSST.csh has done its job.

\item For each package you wish to use, specify which version of that
  package you wish to use by running \texttt{setup [packagename]} on the
  Unix command line, for example \texttt{setup afw}.  Setup is designed to
  help developers switch different versions of, eg, afw in and out for
  testing.  Most readers of this document will not wish to do that,
  but will still need to run setup without specifying any particular
  version; this automatically pulls in the latest version.  You can
  test this step by again entering Python in interactive mode and
  typing \texttt{import lsst.afw}.  Note that the package names you specify
  to setup are less hierarchical than the corresponding Python or C++
  namespaces.  For example, you would \texttt{setup meas\_astrom} and then
  \texttt{import lsst.meas.astrom}.  You do {\it not} \texttt{setup meas::astrom} or 
  \texttt{setup meas.astrom}.

\end{enumerate}

\section{Routinely updating your LSST Software Stack}

The pseudo-package LSSTPipe contains everything you need to run the pipeline; you can re-install at any time to get the latest versions of required pipeline packages, and declare them as current (the default version to be used):
\texttt{lsstpkg install --current LSSTPipe}

When installation is complete, you must run \texttt{setup} on each newly installed package:
\texttt{setup afw}
Appending \texttt{--current} to the end declares your freshly installed version of \texttt{afw} as current:
\texttt{setup afw --current} 

If you wish to setup a different version which is already installed but not declared current:
\texttt{setup afw 3.5.2} 

NOTE: If there is no revision declared current at all, \texttt{setup} will fail and print a message that it doesn't have a current version of that package.

You will most likely find it helpful to understand what the digits in the version numbers mean:
\begin{verbatim}
3.0.1
    |
\end{verbatim}
Indicates minor revisions, such as bug fixes, which can happen quickly and often.
\begin{verbatim}
3.0.1
  |
\end{verbatim}
If this number changes, everything that depends on the package must be upgraded as well.



\section{Update after a critical bug fix}
Sometimes, after a critical bug fix, you might need this latest version before it is available on the distribution server, which means that re-installing LSSTPipe will not deliver the most recent revision.  Here are some notes on what to do in these circumstances.

\begin{itemize}
\item To view all revisions of a package: \texttt{lsstpkg list afw}

\item To list package versions installed on your local machine; this command lists all tags, such as \texttt{Current} or \texttt{Setup}, associated with that particular package:  \texttt{eups list afw}

\item To install a new package, or a new revision of a package, say afw 3.6.0: \texttt{lsstpkg install afw 3.6.0}
\begin{itemize}
\item To also declare this revision as current:  \texttt{lsstpkg install --current afw 3.6.0}
\end{itemize}
\end{itemize}
 
\vskip0.5in

Now you are ready to try the hello world examples in
Chapter~\ref{chap-hello}! 
